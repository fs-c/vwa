\chapter*{Konklusion}
\addcontentsline{toc}{chapter}{Konklusion}

\prettyref{cha:definition-funktion-algorithmen} legt mit einer Definition des Algorithmusbegriffs, Methoden zur Algorithmusbeschreibung und einer einleitenden Definition des Effizienzbegriffs im Kontext die Basis für die darauffolgenden Kapitel.

Mit der $O$-Notation und der dahinführenden asymptotischen Analyse wird in \prettyref{cha:asymptotic-analysis} ein Weg gegeben, die theoretische Effizienz eines Algorithmus zu ermitteln und darzustellen. Eine solche Ermittlung erfolgt in \prettyref{cha:algorithmen} neben der Beschreibung der im Folgenden verwendeten Algorithmen.

Eine konkrete Möglichkeit zur Bestimmung der praktischen Effizienz eines Sortieralgorithmus wird in \prettyref{cha:praktische-effizienz} erarbeitet. Aufbauend auf einer einfachen Testumgebung und einer Strategie zur Funktionsermittlung kann mithilfe des dargelegten \enquote{benchmark}-Programms automatisiert die praktische Effizienz von diversen Algorithmen auf diversen Eingabemengearten ermittelt werden.

\prettyref{cha:vergleich} beschreibt das Verhältnis zwischen praktischer und theoretischer Effizienz anhand der Approximierbarkeit der praktischen aus der theoretischen Effizienz. In \prettyref{sec:approximation} wird dargelegt, dass für die meisten Algorithmen ein einziger Datenpunkt und Kenntnis der theoretischen Effizienz für eine praktikable Approximation der praktischen Effizienz ausreicht. \prettyref{sec:best-algo} beschreibt die Schwierigkeit der Ermittlung eines \enquote{besten} Algorithmus im Allgemeinen, und legt einen Weg zur Ermittlung eines \emph{guten} Algorithmus für einen gewissen Einsatzbereich dar.
