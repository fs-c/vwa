\chapter*{Schluss}
\addcontentsline{toc}{chapter}{Schluss}

% \prettyref{cha:definition-funktion-algorithmen} legt mit einer Definition des Algorithmusbegriffs, Methoden zur Algorithmusbeschreibung und einer einleitenden Definition des Effizienzbegriffs im Kontext die Basis für die darauffolgenden Kapitel.

% Mit der $O$-Notation und der dahinführenden asymptotischen Analyse wird in \prettyref{cha:asymptotic-analysis} ein Weg gegeben, die theoretische Effizienz eines Algorithmus zu ermitteln und darzustellen. Eine solche Ermittlung erfolgt in \prettyref{cha:algorithmen} neben der Beschreibung der im Folgenden verwendeten Algorithmen.

% Eine konkrete Möglichkeit zur Bestimmung der praktischen Effizienz eines Sortieralgorithmus wird in \prettyref{cha:praktische-effizienz} erarbeitet. Aufbauend auf einer einfachen Testumgebung und einer Strategie zur Funktionsermittlung kann mithilfe des dargelegten \enquote{benchmark}-Programms automatisiert die praktische Effizienz von diversen Algorithmen auf diversen Eingabemengearten ermittelt werden.

% \prettyref{cha:vergleich} beschreibt das Verhältnis zwischen praktischer und theoretischer Effizienz anhand der Approximierbarkeit der praktischen aus der theoretischen Effizienz. In \prettyref{sec:approximation} wird dargelegt, dass für die meisten Algorithmen ein einziger Datenpunkt und Kenntnis der theoretischen Effizienz für eine praktikable Approximation der praktischen Effizienz ausreicht. \prettyref{sec:best-algo} beschreibt die Schwierigkeit der Ermittlung eines \enquote{besten} Algorithmus im Allgemeinen, und legt einen Weg zur Ermittlung eines \emph{guten} Algorithmus für einen gewissen Einsatzbereich dar.

Die zwei Leitfragen dieser Arbeit wurden insofern beantwortet als
\begin{itemize}
    \item \prettyref{cha:asymptotic-analysis} und \prettyref{cha:praktische-effizienz} Definitionen und Ermittlungswege für die theoretische und praktische Effizienz bieten.
    \item \prettyref{cha:vergleich} das Verhältnis zwischen diesen Effizienzen entlang zweier Blickwinkel behandelt und dargestellt hat.
\end{itemize}

In \prettyref{cha:definition-funktion-algorithmen} wurde ein Algorithmus als ein wohldefiniertes, endliches und effektives Berechnungsverfahren mit Eingaben und Ausgaben definiert, eine bestgeeignete Darstellungsform erarbeitet und eine Definition des Effizienzbegriffs vorgenommen. Genauer wurde die Effizienz als Funktion $T$ in Abhängigkeit der Eingabegröße definiert, wobei die Funktionswerte die Laufzeit darstellen.

\prettyref{cha:asymptotic-analysis} leitete die asymptotische Analyse mit dem Ziel der Ermittlung der Größenordnung des Wachstums -- der Komplexität -- einer Funktion ein. Weiters wurde die theoretische Effizienz eines Algorithmus als die Komplexität der \enquote{Effizienzfunktion} $T$ aus \prettyref{cha:definition-funktion-algorithmen} definiert, wobei die Funktionswerte anstelle einer konkreten Laufzeit die Anzahl getätigter einfachen Operationen repräsentieren. Für die Darstellung der theoretischen Effizienzen wurde die $O$-Notation eingeführt.

In \prettyref{cha:praktische-effizienz} wurde die praktische Effizienz als Zeit, die ein Algorithmus für das bearbeiten einer gewissen Eingabemenge benötigt, formuliert und ein konkretes Programm zur Ermittlung dieser praktischen Effizienz beleuchtet. Teil davon war eine Implementation von Generatoren für diverse Eingabemengen, eine Testumgebung zur Messung der Laufzeit und Mechanismen zur automatisierten Generierung.

\prettyref{cha:vergleich} zeigte das Verhältnis zwischen praktischer und theoretischer Effizienz anhand der Approximierbarkeit der praktischen aus der theoretischen Effizienz. In \prettyref{sec:approximation} wurde dargelegt, dass für die meisten Algorithmen ein einziger Datenpunkt und Kenntnis der theoretischen Effizienz für eine praktikable Approximation der praktischen Effizienz ausreicht. \prettyref{sec:best-algo} zeigte die Schwierigkeit der Ermittlung eines \enquote{besten} Algorithmus im Allgemeinen, und legte einen Weg zur Ermittlung eines \emph{guten} Algorithmus für einen gewissen Einsatzbereich dar.

Hier wurde die wichtigste Einschränkung der theoretischen Effizienz erkenntlich: Für kleine $n$ ist sie nur sehr bedingt für eine Einschätzung der relativen Qualität von Algorithmen geeignet. Ebenfalls wurde hier offensichtlich, dass die $O$-Notation (also das verwendete Werkzeug zur Darstellung und Ermittlung der theoretischen Effizienz) keine Aussage über ihre Qualität bzw. ihre Präzision macht.

So wiesen etwa drei Algorithmen mit gleicher theoretischer Effizienz im günstigsten Fall -- Heap-, Merge- und Quicksort -- selbst bei großen Eingabemengen stark unterschiedliche Werte auf (\prettyref{subfig:results-boxplots-md-sorted}). Ein noch deutlicheres Beispiel war der Quicksort, dessen theoretische Effizienz im ungünstigsten Fall bedeutend schlechter als jene von Heap- und Mergesort, und gleich jener des Insertionsort war. Dennoch war er konsistent der schnellste Algorithmus auf invers sortierten Listen (\prettyref{subfig:results-boxplots-xs-inverted}, \ref{subfig:results-boxplots-sm-inverted} und \ref{subfig:results-boxplots-md-inverted}).
