\chapter{Laufzeitermittlung als Effizienzangabe}

In diesem Kapitel wird eine Methode zur Ermittlung der Laufzeit von Algorithmen dargestellt.

Es gilt die Laufzeit eines Algorithmus zu ermitteln und diese annähernd als Funktion $T(n)$ (wobei $T(n)$ die Zeit und $n$ die Eingabegröße ist) darzustellen. Das Ziel deckt sich also mit jenem der asymptotschen Analyse aus \prettyref{cha:asymptotic-analysis}. Genauer gilt es, ebenfalls wie in der asymptotischen Analyse, eine solche Funktion für diverse Arten von Eingabemengen zu ermitteln (\cite[27]{mcg2012}).

Nach \cite[10]{mcg2012} kann \enquote{the experimental process} im Kontext der empirischen Algorithmusanalyse im Wesentlichen grob in folgende vier Schritte aufgegliedert werden.

\begin{enumerate}
    \item Formuliere eine Frage.\label{itm:experiment-formulate-question}
    \item Stelle eine Testumgebung bereit.\label{itm:experiment-environment}
    \item Gestalte ein Experiment welches die Frage aus \prettyref{itm:experiment-formulate-question} anspricht.
    \item Führe das Testprogramm aus und sammle die Daten.
\end{enumerate}

Eine grobe Formulierung der Frage ergibt sich schon aus der vorhergehenden Einleitung dieses Kapitels:

\spaced{\enquote{Wie viel Zeit benötigt ein Sortieralgorithmus um Eingabemengen verschiedener Art und Größe in eine sortierte Ausgabemenge zu überführen?}}

Um diese Frage umsetzbar zu machen, gilt es nunmehr nur noch die verschiedenen Arten und Größen der Eingabemengen zu definieren, dies geschieht in \prettyref{sec:runtime-inputs}.

Die Testumgebung aus \prettyref{itm:experiment-environment} wird in \prettyref{sec:runtime-environment} näher definiert. 

\ifndef{vwarelease}{
\emph{Hier sind \citetitle{joh2002} (\cite{joh2002}) und \citetitle{mcg2012} (\cite{mcg2012}) wichtig. \cite[83]{sha2011} kann als Argumentation für das ausgeprägte Verwenden der \emph{Standard Library} ausgelegt werden nachdem hier potentielle Vorurteile bzw. Ungleichheiten bei der Programmierung de facto wegfallen.}

\emph{Der Werdegang der Empirie in den Computerwissenschaften ist nicht unspannend: Zu Zeiten von \cite{joh2002} war die empirische Analyse als Feld offenbar noch bei weitem nicht so weit fortgeschritten und verbreitet wie im Erscheinungsjahr von \cite{mcg2012}. Beide beinhalten weitgehend äquivalente Kernaussagen, aber erstere Publikation (ein Journalartikel) ist noch bei weitem unausgereifter als letztere (ein Buch). 
}}{}

\section{Eingabemengen}
\label{sec:runtime-inputs}

Analog 

\section{Testumgebung}
\label{sec:runtime-environment}


