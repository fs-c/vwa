\chapter{Laufzeitermittlung als Effizienzangabe}
\label{cha:praktische-effizienz}

In diesem Kapitel wird eine Methode zur Ermittlung der Laufzeit von Algorithmen dargestellt.

\prettyref{sec:runtime-question} beschäftigt sich, als Ausgangspunkt dieses Kapitels, mit der Aufstellung einer Frage und einer Konkretisierung des Ziels: Der Ermittlung einer \enquote{praktischen Effizienz}.

Um diese Frage zu konkretisieren und umsetzbar zu machen, werden in \prettyref{sec:runtime-inputs} Arten und Größen der Eingabemengen definiert. \prettyref{sec:runtime-environment} behandelt die Bereitstellung einer Testumgebung.

Ein konkreter Weg zur \enquote{praktischen Effizienz} wird in \prettyref{sec:funkterm} gegeben, \prettyref{sec:runtime-orchestration} behandelt einen Konstrukt zur Orchestrierung dieser Vorgehensweise.

Für konkrete Implementationen und Quellcode-Beispiele wird die Programmiersprache C++ (\cite{ISO-C++17}, \enquote{C++17}) verwendet.

\ifndef{vwarelease}{
\emph{Hier sind \citetitle{joh2002} (\cite{joh2002}) und \citetitle{mcg2012} (\cite{mcg2012}) wichtig. \cite[83]{sha2011} kann als Argumentation für das ausgeprägte Verwenden der \emph{Standard Library} ausgelegt werden nachdem hier potentielle Vorurteile bzw. Ungleichheiten bei der Programmierung de facto wegfallen.}

\emph{Der Werdegang der Empirie in den Computerwissenschaften ist nicht unspannend: Zu Zeiten von \cite{joh2002} war die empirische Analyse als Feld offenbar noch bei weitem nicht so weit fortgeschritten und verbreitet wie im Erscheinungsjahr von \cite{mcg2012}. Beide beinhalten weitgehend äquivalente Kernaussagen, aber erstere Publikation ist noch bei weitem unausgereifter als letztere. 
}}{}

\section{Formulierung einer Frage}
\label{sec:runtime-question}

Es gilt die Laufzeit eines Algorithmus zu ermitteln und diese annähernd als Funktion $T(n)$ (wobei $T(n)$ die Zeit und $n$ die Eingabegröße ist) darzustellen. Das Ziel deckt sich also mit jenem der asymptotischen Analyse aus \prettyref{cha:asymptotic-analysis}. Genauer gilt es, ebenfalls wie in der asymptotischen Analyse, eine solche Funktion für diverse Arten von Eingabemengen zu ermitteln (\cite[27]{mcg2012}).

Nach \cite[10]{mcg2012} kann \enquote{the experimental process} im Kontext der empirischen Algorithmusanalyse im Wesentlichen grob in die Formulierung einer Frage, die Bereitstellung einer die Frage behandelnden Testumgebung bzw. eines Testprogramms und die Ausführung des Testprogramms aufgegliedert werden.

Eine grobe Formulierung der Frage ergibt sich schon aus der vorhergehenden Einleitung dieses Kapitels: Wie viel Zeit benötigt ein Sortieralgorithmus um Eingabemengen verschiedener Art und Größe in eine sortierte Ausgabemenge zu überführen?

\section{Eingabemengen}
\label{sec:runtime-inputs}

Für die Ermittlung der praktischen Effizienz werden sortierte, invertierte und zufällig geordnete Eingabemengen verwendet. Um diese Mengen zu generieren, werden die Funktionen \lstinline{sets::sorted}, \lstinline{sets::inverted} und \lstinline{sets::random} verwendet.

\begin{lstlisting}[caption={Funktionen, welche Mengen der verwendeten Eingabemengearten generieren.}]
namespace sets {
    using set_t = std::vector<int>;

    // ...
}

sets::set_t sets::sorted(const size_t size) {
	auto set = set_t(size);

	std::iota(set.begin(), set.end(), 1);

	return set;
}

sets::set_t sets::inverted(const size_t size) {
	auto set = set_t(size);

	std::iota(std::rbegin(set), std::rend(set), 1);

	return set;
}

sets::set_t sets::random(const size_t size) {
	auto set = sorted(size);

	utils::random_shuffle(set.begin(), set.end());

	return set;
}
\end{lstlisting}

(Die Funktion \lstinline{utils::random_shuffle}, welche für das Generieren der zufällig geordneten Menge verwendet wird, verwendet den Mersenne-Twister-Algorithmus, um hochwertige Pseudozufallszahlen zu generieren und damit die Eingabemenge \enquote{durchzumischen}. Siehe \prettyref{lst:random-shuffle} für die verwendete Implementation der Funktion.)

Eine ausführlichere Auswahl an Eingangsmengen, welche an einen sie bearbeitenden Algorithmen angepasst ist -- wie sie für eine eingehende Analyse eines einzelnen Algorithmus angebracht wäre (vgl. \cite[27]{mcg2012}) --, würde das gesetzte Ziel dieser Arbeit verfehlen.

\section{Testumgebung}
\label{sec:runtime-environment}

Es gilt die Zeit, welche eine Funktion für die vollständige Ausführung benötigt, zu messen. Die Funktion wird ausgeführt, die Zeit unmittelbar vor (\id{start}) und nach (\id{end}) der Ausführung wird gemessen. Die Laufzeit des Algorithmus ist nun gleich $\id{end} - \id{start}$.

Eine konkrete Implementation einer Klasse zur Messung der Laufzeit eines Algorithmus ist in \prettyref{lst:experiment} gegeben.

\begin{lstlisting}[caption={Implementation einer Klasse zur Ermittlung der Laufzeit eines Algorithmus.}, label=lst:experiment]
class experiment {
	std::function<void()> function;

	using clock_t = std::chrono::high_resolution_clock;

public:
	using duration_t = clock_t::duration;

	explicit experiment(std::function<void()> function)
		: function(std::move(function)) {};

	[[nodiscard]] duration_t run() const;
};


experiment::duration_t experiment::run() const {
	const auto start = clock_t::now();

	function();

	const auto end = clock_t::now();

	return duration_t{end - start};
}
\end{lstlisting}
\noindent
Die Laufzeit eines einfachen Algorithmus kann mit ihrer Hilfe durch
\begin{lstlisting}[numbers=none]
void a1() { ... }
const auto time = experiment(a1).run();
\end{lstlisting}
ermittelt werden, wobei die Zeitspanne in der für das System kleinstmöglichen Einheit angegeben ist (vgl. \cite[652]{ISO-C++17}). Auf dem verwendeten Testsystem wird die Zeit in Nanosekunden gemessen, $1ns = 10^{-9}s = 0,000000001s$.

\paragraph{Algorithmen mit Eingabewerten} Der Konstruktor in der Klasse aus \prettyref{lst:experiment} erwartet als einzige Eingabe eine Variable des Typs \lstinline{std::function<void()>}, also eine Funktion ohne Eingabe- und Rückgabewerte. Dies ist der Fall, um größtmögliche Flexibilität in der Verwendung der Testumgebung zu gewährleisten.

Die in dieser Arbeit behandelten Algorithmen haben alle einen oder mehrere Eingabewerte, sie erfüllen also nicht die Form, wie sie vom Konstruktor erwartet wird. Um ein Experiment mit einer Funktion eines anderen Typs als \lstinline{void()} zu initialisieren, wird ein \enquote{\emph{wrapper}} verwendet.

\begin{minipage}{\linewidth}
\begin{lstlisting}[numbers=none]
const size_t size = 1337;
void |\id{func}|(const size_t s) { ... }
const auto |\id{wrapper}| = [&]() { func(size); };
const auto time = experiment(wrapper).run();
\end{lstlisting}
\end{minipage}

Im obenstehenden Beispielcode wird ein Lambda-Ausdruck verwendet, um eine \enquote{umwickelnde Funktion}, einen sogenannten \id{wrapper}, zu erstellen. Ein Aufruf von \id{wrapper} führt zum Aufruf der Funktion \id{func} mit dem Eingabeparameter \lstinline{size} bzw. 1337.

\section{Ermittlung der praktischen Effizienz}
\label{sec:funkterm}

Das Ziel ist es, eine Funktion aufzustellen, bzw. zu approximieren, welche die Laufzeit eines Algorithmus in Abhängigkeit der Eingabegröße darstellt (vgl. \cite[37]{mcg2012}). Um diesem Ziel nachzukommen, gilt es, die Eingabemenge in wachsende Untermengen aufzuteilen, diese Untermengen zu sortieren und die dafür benötigte Zeit (in Kombination mit der Größe der jeweiligen Untermenge) auszugeben.

Hierfür ist eine Klasse \lstinline{class benchmark} gegeben, die mit einer Eingabemenge, einem Sortieralgorithmus, einer Angabe zur Art der Ermittlung der Untermengen und der Anzahl der zu erstellenden Untermengen konstruiert wird.

\begin{lstlisting}[caption={Klasse zur Approximation einer Funktion der Laufzeit eines Algorithmus in Abhängigkeit der Eingabegröße.}, label=lst:simple-benchmark]
class benchmark {
public:
	using algorithm_t = sorters::sorter_t<sets::iterator_t>;

	struct result : public std::map<size_t, experiment::duration_t> {
	    // ...
	}

	// ...

	benchmark(sets::set_t set, algorithm_t algorithm, benchmark::step_type_t step_type, size_t total_chunks);

	[[nodiscard]] benchmark::result run() const;
};
\end{lstlisting}
\noindent
In \prettyref{lst:simple-benchmark} wird ein vereinfachter Einblick in eine Schnittstelle zu einem, diese Anforderungen erfüllenden, Mechanismus gegeben\footnote{Für eine vollständige Implementation siehe \crBenchmark~und \crBenchmarkImpl.}.

Die Funktion \lstinline{benchmark::run} retourniert eine Instanz der Klasse \lstinline{class benchmark::result}, die wiederum eine Menge von Größenangaben und damit assoziierten Zeitangaben ist. Eine von dieser Funktion zurückgegebene Menge von Paaren hat beispielsweise die Form:
\begin{equation*}
	\{
		(1, 0.0768125),
		(2, 0.104375),
		(3, 0.129938),
		(4, 0.170375),
		\ldots,
		(n, t)
	\}
\end{equation*}
Sie ist so zu interpretieren, dass der gegebene Algorithmus für das Sortieren einer Untermenge der Größe $n$ der gegebenen Menge $t$ Zeiteinheiten benötigt hat.

\paragraph{Art der Schrittgröße}

In \lstinline{benchmark.run} erfolgt also eine Teilung der ursprünglichen Eingabemenge in eine gewisse Anzahl von Untermengen. Die Art der Wachstumsrate der Größe dieser Untermengen wird im Konstruktor der Klasse festgelegt, wo entweder ein lineares oder ein quadratisches Wachstum gewählt werden kann (siehe \prettyref{fig:step-type-graphs}).

\begin{figure}[h]
	\centering
	\begin{subfigure}[T]{0.455\textwidth}
		\begin{tikzpicture}
			\begin{axis}[
				default, width=0.95\textwidth,
				xmin=0, ymin=0, xmax=32,
				domain=0:32,
				legend pos=north west,
				xlabel={Index},
				ylabel={Größe},
			]
				\addplot[black] {(262144 / 128) * x};
				\addlegendentry{$l$};
				\addplot[black, thick] {262144 / 128^2) * x^2};
				\addlegendentry{$q$};
			\end{axis}
		\end{tikzpicture}
	\end{subfigure}
	\hfill
	\begin{subfigure}[T]{0.455\textwidth}
		\begin{tikzpicture}
			\begin{axis}[
				default, width=0.95\textwidth,
				xmin=0, ymin=0, xmax=128,
				domain=0:128,
			]
				\addplot[black] {(262144 / 128) * x};
				\addplot[black, thick] {262144 / 128^2) * x^2};
			\end{axis}
		\end{tikzpicture}
	\end{subfigure}
	\caption{Graphen der Untermengengrößen bei einer gesamten Mengengröße von $2^{18}$, geteilt auf 128 Untermengen mit linearer ($l$) und quadratischer ($q$) Schrittgröße.}
	\label{fig:step-type-graphs}
\end{figure}


\section{Orchestrierung}
\label{sec:runtime-orchestration}

In den vorhergehenden Abschnitten wurde

\begin{itemize}
    \item eine Definition für einige Eingabemengen mit ihren korrespondierenden Generatoren gegeben (\prettyref{sec:runtime-inputs}).
    \item eine Testumgebung (\lstinline{class experiment}) zur Ermittlung der Laufzeit eines Algorithmus mit einer bestimmten Eingabemenge und Größe (\prettyref{sec:runtime-environment}).
    \item einen Mechanismus zur Laufzeitermittlung für steigende bzw. variierende Eingabegrößen (\lstinline{class benchmark}, \prettyref{sec:funkterm}).
\end{itemize}

Noch abgehend ist ein Mechanismus zur \emph{Orchestrierung} bzw. \emph{Instrumentalisierung} der bestehenden Teile. Genauer gilt es, ein ausführbares Programm bereitzustellen, welches mithilfe der \lstinline{benchmark}-Klasse für alle möglichen Kombinationen aus den Algorithmen und Eingabearten Benchmarkresultate ermittelt und diese ausgibt. Ebenfalls gilt es, einen Mechanismus zur einfachen Kontrolle des Umfangs und der Präzision dieser Messungen bereitzustellen.

\begin{figure}[htp]
    \centering
    \begin{subfigure}[b]{0.40\textwidth}
        \centering
        \begin{forest}
            for tree={grow'=0,folder}
            [out/
                [heap\_inverted]
                [heap\_random]
                [heap\_sorted]
                [insertion\_inverted]
                [\ldots]
            ]
        \end{forest}
        \caption{Struktur des Ausgabeordners, die Endung \enquote{/} repräsentiert einen Ordner.}
    \end{subfigure}
    \hfill
    \begin{subfigure}[b]{0.40\textwidth}
        \centering
        \begin{tabular}{c c}
            $n$ & $t$ in $\mu s$ \\
            \midrule
            2048 & 1425.91 \\
            4096 & 3702.23 \\
            6144 & 4790.03 \\
            8192 & 6419.75 \\
            10240 & 8436.86 \\
            \vdots & \vdots \\
        \end{tabular}
        \caption{Inhalt der ersten 5 Zeilen der Datei heap\_inverted.}
    \end{subfigure}
    \caption{Beispielhafte Ordner- und Dateistruktur nach Ausführung des Programms zur Benchmark--Orchestrierung.}
    \label{fig:benchmark-file-structure}
\end{figure}


Diese Anforderungen erfüllt das benchmark-Programm bzw. der Code in \crMain. Standardmäßig erstellt es einen Ordner \emph{out} und befüllt diesen mit gesammelten Daten. Für jede mögliche Kombination aus Algorithmus und Eingabeart wird eine Tabelle mit Untermengengrößen und Zeitangaben erstellt, siehe \prettyref{fig:benchmark-file-structure}.

Eine Steuerung der Vorgänge zur Datensammlung und -aufbereitung im Programm kann durch Parameter auf der Kommandozeile erfolgen. So sagt etwa der Befehl \enquote{\lstinline{benchmark -o data -s 512}} aus, dass ein Ordner \emph{data} (anstelle von out) erstellt werden soll. Ebenfalls soll die größte zu sortierende Liste genau 512 Elemente haben. Alle möglichen Befehle sind \prettyref{fig:benchmark-usage} zu entnehmen.

\begin{figure}[hbp]
    \begin{lstlisting}[language=none, numbers=none]
Usage: benchmark [options]
Options:
  -h, --help        Display this information and exit.
  -o, --output      Sets the output path. (default: ./out)
  -s, --size        Sets the sample size. (default: 262144 i.e. 2^18)
  -c, --chunks      Sets the number of chunks that the set will be devided into.
                    (default: 128)
  -t, --step-type   Specifies the step type to use, one of 'linear' or
                    'quadratic'. (default: linear)
  -a, --average     Repeat the benchmarking a given number of times and output
                    the average results of all runs. (default: 0)
  -m, --median      Like -a, except it outputs the median. (default: 0)
  -r, --randomize   Randomize the order in which the different combinations
                    between sorters and sets are benchmarked. (default: false)
    \end{lstlisting}
    \caption{Anwendungsbeschreibung des benchmark-Programms, ausgegeben nach Ausführung von \lstinline{benchmark -h}.}
    \label{fig:benchmark-usage}
\end{figure}

% \paragraph{Ergebnisverfälschung durch wiederholtes Ausführen}

% Es kann wünschenswert sein ein Experiment wiederholt auszuführen und den Durchschnitt oder Median der Ergebnisse zu betrachten (vgl. \cite[20]{mcg2012}), etwa um experimentunabhängige Schwankungen in der Ausführungsgeschwindgkeit des Prozessors \enquote{auszuglätten}. Im benchmark-Programm ist das konkret jeweils durch die Parameter \lstinline{-a} bzw. \lstinline{--average} und \lstinline{-m} bzw. \lstinline{--median} umsetzbar.

% Hier ist allerdings Acht geboten: Wiederholte Ausführung eines Programms kann zu einer Beschleunigung der Ausführung führen. \prettyref{fig:repeated-distortion-simple} zeigt etwa eine erhebliche  

% \begin{figure}[htp]
%     \centering
% 	\begin{tikzpicture}
%         \begin{axis}[
%             default,
%             legend pos=north west, xlabel={Größe}, ylabel={Zeit},
%         ]
%             \addplot [black, dashed] table {data/supplementary/distortion/1/quick_random};
%             \addlegendentry{1};

%             \addplot [black] table {data/supplementary/distortion/8/quick_random};
%             \addlegendentry{8};

%             \addplot [black, thick] table {data/supplementary/distortion/128/quick_random};
%             \addlegendentry{128};

%             \addplot [black, very thick] table {data/supplementary/distortion/512/quick_random};
%             \addlegendentry{512};
%         \end{axis}
%     \end{tikzpicture}
%     \caption{Ergebnis des benchmark-Programms für den Quicksort Algorithmus auf einer zufälligen Eingabenge. Die Legende repräsentiert die $n$te Ausführung.}
% 	\label{fig:repeated-distortion-simple}
% \end{figure}

% In \prettyref{fig:sequential-runs} ist erkenntlich, dass mehrfache sequentielle Ausführung des Benchmarks die benötigte Zeit im gegeben Fall erheblich reduziert. Um die praktische Relevanz der konkreten experimentellen Ergebnisse bzw. Messwerte zu gewährleisten\footnote{Nachdem eine derartige wiederholte Ausführung in der Praxis unwahrscheinlich ist (vgl. \cite[5]{taocp3}).} gilt es also diese Verfälschung unter Aufrechterhaltung der Vorteile der wiederholten Ausführung zu eliminieren bzw. zu reduzieren. 
