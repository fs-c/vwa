\chapter*{Abstract}
\label{cha:abstract}

Diese Arbeit beschäftigt sich mit der Ermittlung der \enquote{theoretischen-} und der \enquote{praktische Effizienz} eines Algorithmus, und dem Verhältnis der theoretischen zur praktischen Effizienz, am Beispiel von Sortieralgorithmen. Dahingehend werden einige Sortieralgorithmen mithilfe asymptotischer Analyse und der $O$-Notation bezüglich ihrer theoretischen Effizienz, und mithilfe eines maßgeschneiderten benchmark-Programms hinsichtlich ihrer praktischen Effizienz untersucht. Darauf aufbauend wird festgestellt, dass die Kenntnis der theoretischen Effizienz und eines einzigen Datenpunktes für eine praktikable Approximation der praktischen Effizienz ausreicht. Weiters geht aus einer Analyse der praktischen Effizienzen der betrachteten Algorithmen und Eingabemengen der Insertionsort als bester Algorithmus für kleine Mengen, und der Quicksort als bester Algorithmus für große Mengen hervor.
