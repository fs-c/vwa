\chapter{Definition und Funktion von Algorithmen}
\label{cha:definition-funktion-algorithmen}
% PB1
\vspace{-0.25cm}
Dieses Kapitel beschäftigt sich mit der Definition des Algorithmus- und, in diesem Kontext, Effizienzbegriffs, sowie mit der grundlegenden Funktionsweise von Algorithmen am Beispiel von Sortieralgorithmen.

In \prettyref{sec:algorithm-definition} werden der Ursprung und vorhergehende Bedeutungen des Wortes \enquote{Algorithmus} behandelt. Ebenso wird eine, zur kontemporären Bedeutung des Begriffs passende, Begriffsdefinition aufgestellt.

Ein Überblick über die verschiedenen Möglichkeiten zur Darstellung von Algorithmen wird in \prettyref{sec:algorithm-specification-overview} gegeben. Die in der folgenden Arbeit verwendete Art der Beschreibung wird in \prettyref{sec:alg-spec-pseudo} im Detail beschrieben.

Die Begrifflichkeit \enquote{Sortieralgorithmus} wird in \prettyref{sec:sorting-algorithms-definition} definiert. Anhand eines Beispielalgorithmus, dem \text{Insertion-Sort}, wird die Arbeitsweise von Algorithmen, und insbesondere von Sortieralgorithmen, dargestellt.

In \prettyref{sec:algo-effizienz} wird eine Definition des Effizienzbegriffs im Kontext von Algorithmen vorgenommen.
%
% To fix page break 1
\vspace{-0.25cm}
\section{Definition}
\label{sec:algorithm-definition}

Noch um 1957 war \enquote{\emph{algorithm}} nicht in \emph{Webster's New World Dictionary} vertreten. Dort war nur der ältere Begriff \enquote{\emph{algorism}} zu finden, zu Deutsch das Rechnen mit der arabischen Zahlschrift. \emph{Algorism} entstammt dem Namen eines persischen Autors, \emph{Abū 'Abd Allāh Muhammad ibn Mūsā al-Khwārizmī} \footnotemark (circa 825 A. D., vgl. \cite[1--2]{taocp1}). Al-Khwārizmī schrieb das Buch \emph{Kitab al jabr wa'l-muqābala} (\enquote{[\ldots] Rechenverfahren durch Ergänzen und Ausgleichen}), aus dessen Titel ein anderes Wort, \enquote{Algebra}, entstammt (\cite[197--199]{ger1984}). Die Wandlung des Begriffs \emph{algorism} zu \emph{algorithm} (und damit Algorithmus) ist in \emph{Vollständiges mathematisches Lexicon} (\cite[38]{wol1747}) dokumentiert, hier wird der Begriff wie folgt definiert: \enquote{Unter dieser Benennung werden zusammen begriffen die 4 Rechnungs-Arten in der Rechen-Kunst nemlich addiren, multipliciren, subtrahiren und dividiren. [...]} (alte Rechtschreibung übernommen).

\footnotetext{Al-Khwārizmīs Name stellt in der Literatur eine Quelle der Verwirrung dar: \cite[1]{hsr1997}, \cite[1]{taocp1}, \cite[84]{pic2009} und \cite[197--199]{ger1984} verwenden bei ihren Erwähnungen jeweils unterschiedliche Namen, sprechen jedoch von derselben Person.}

Vor 1950 wurde der Begriff Algorithmus am häufigsten mit dem euklidischen Algorithmus assoziiert, einem Verfahren zur Ermittlung des größten gemeinsamen Teilers zweier Zahlen (vgl. \cite[2]{taocp1}).

Die moderne Bedeutung des Wortes \emph{Algorithmus} ist nicht unähnlich zu den Bedeutungen der Wörter \emph{Rezept}, \emph{Vorgang}, \emph{Verfahren}, et cetera. Dennoch konnotiert das Wort etwas anderes -- es ist nicht nur eine endliche Menge von Regeln, die eine Folge von Operationen für die Lösung einer bestimmten Aufgabe darstellen, ein Algorithmus hat nach \cite[1]{hsr1997} zwingend folgende fünf Merkmale:

\begin{enumerate}
    \item \emph{Eingabe}. Ein Algorithmus hat keine, eine oder mehrere Eingangsmengen, welche entweder zu Beginn oder während der Laufzeit gegeben werden.

    \item \emph{Ausgabe}. Ein Algorithmus hat eine oder mehrere Ausgabemengen, welche eine wohldefinierte Beziehung zu den Eingangsmengen haben.
    
    \item \emph{Eindeutigkeit}. Jeder Arbeitsschritt eines Algorithmus ist eindeutig definiert.

    \item \emph{Endlichkeit}. Ein Algorithmus terminiert\footnote{Abbruch mit Erfolg oder Misserfolg} nach einer endlichen Anzahl von Arbeitsschritten.

    \item \emph{Effektivität}. Die Arbeitsschritte müssen einfach genug sein, um in endlicher Zeit von einer Person mit Stift und Papier ausgeführt werden zu können. \ifndef{vwarelease}{\emph{Mit Beispielen besser zu erklären.}}{}
\end{enumerate}

Etwas informeller kann ein Algorithmus als ein \emph{wohldefiniertes} (vgl. \cite[16]{alu2009}) Berechnungsverfahren (engl. \emph{computational procedure}) beschrieben werden, das einen Wert, oder eine Menge von Werten, als Eingabe erhält und einen Wert, oder eine Menge von Werten, als Ausgabe produziert (vgl. \cite[5]{clrs2001}). Ein Algorithmus ist also eine Folge von Berechnungsschritten, welche Eingabe zu Ausgabe überführen.

\section{Methoden zur Algorithmusbeschreibung}
\label{sec:algorithm-specification-overview}

Algorithmen können auf vielfältige Weise beschrieben und dargestellt werden.

\citeauthor{taocp1} verwendet etwa in seinem Standardwerk \enquote{\citetitle{taocp1}} (\cite{taocp1}) eine Mischung aus natürlicher Sprache und mathematischen Ausdrücken. 

\begin{figure}[hbt]
    \begin{algo}{E}{Euklids Algorithmus}
        Gegeben seien zwei positive ganze Zahlen $m$ und $n$. Der \emph{größte gemeinsame Teiler}, also die größte positive ganze Zahl welche sowohl $m$ als auch $n$ gerade teilt, ist zu ermitteln.
    
        \begin{algosteps}
            \algostep[Ermittle den Rest.] Dividiere $m$ durch $n$, $r$ ist der Rest. (Es gilt nun also $0 \leq r < n$.)\label{step:euklid1}
    
            \algostep[Ist er Null?] Gilt $r = 0$ so terminiert der Algorithmus, $n$ ist die Lösung.\label{step:euklid2}
            
            \algostep[Verringere.] Setze $m \leftarrow n$, $n \leftarrow r$ und gehe zurück zu Schritt \ref{step:euklid1}\label{step:euklid3}.
        \end{algosteps}
    \end{algo}
    \caption{Algorithmusbeschreibung nach \citeauthor{taocp1}. Unter Anderem verwendet in \cite{taocp1} und \cite{taocp3}. Beispiel entnommen aus \cite[2]{taocp1}, Algorithm E (übersetzt aus dem Englischen).}
    \label{fig:knuth-algorithm-specification}
\end{figure}


Diese Art der Algorithmusbeschreibung, wie sie beispielhaft in \prettyref{fig:knuth-algorithm-specification} anhand von Euklids Algorithmus dargestellt wird, gliedert sich gut in den Lesefluss ein, und ist am besten geeignet, um die Funktionsweise eines Algorithmus verständlich darzustellen (\cite[147]{zob2015}).

Eine andere Methode ist die Darstellung eines Algorithmus mithilfe von \emph{Flowcharts}, wie sie in \prettyref{fig:flowchart-algorithm-specification} wieder am Beispiel von Euklids Algorithmus dargestellt wird.

\begin{figure}[hbt]
    \centering
    \makebox[\textwidth][c]{
        \begin{tikzpicture}[
            auto,
            node distance=1.5cm,
            algostep/.style={rectangle, minimum size=6mm, draw=black, inner xsep=0.5em}
        ]
            \node (s1) [algostep] {\textbf{\ref{step:euklid1}}. Finde den Rest.};
            \node (s2) [algostep, rounded corners=3mm, right=of s1] {\textbf{\ref{step:euklid2}}. Ist er Null?};
            \node (s3) [algostep, right=of s2] {\textbf{\ref{step:euklid3}}. Verringere.};

            \node (a1) [above=0.7 cm of s1] {$\{n, m\}$};
            \node (a2) [below=0.7 cm of s2] {$n$};

            \draw [->]
                (s1) edge (s2)
                (s2) edge node {Nein} (s3)
                (a1) edge (s1)
                (s2) edge node {Ja} (a2);
            
            \draw [->] (s3.north) -- +(0,0.5) -| (s1);
        \end{tikzpicture}
    }
    \caption{Algorithmusbeschreibung in Form eines Flowcharts. Beispiel in abgeänderter Form aus \cite[3]{taocp1}, Fig. 1 entnommen.}
    \label{fig:flowchart-algorithm-specification}
\end{figure}


Diese Art der Beschreibung eignet sich am besten für kleine bzw. kurze Algorithmen (\cite[5]{hsr1997}). Sie wird (wie etwa auch von \citeauthor{taocp1}) meistens in Kombination mit einer anderen Algorithmusbeschreibung verwendet, um diese zu komplementieren.

Die wohl am weitreichendsten etablierte Art der Beschreibung von Algorithmen ist jedoch \emph{Pseudocode} (vgl. \cite[147]{zob2015}), wie er etwa in \prettyref{fig:pseudocode-algorithm-specification} zu sehen ist.

\begin{figure}
    \centering
    \begin{codebox}
        \Procname{$\proc{Euklid}(n, m)$}
        \li \While $(r \gets n \mod m) \neq 0$
        \li     \Do
                    $m \gets n$                 \label{ln:euclid-while-begin}
        \li         $n \gets r$                 \label{ln:euclid-while-end}
                \End
        \li \Return $n$                         \label{ln:euclid-return}
    \end{codebox}
    \caption{Algorithmusbeschreibung durch \emph{Pseudocode}.}
    \label{fig:pseudocode-algorithm-specification}
\end{figure}

Pseudocode ist näher zu tatsächlichen Programmiersprachen, in denen ein Algorithmus implementiert werden könnte, als die bisher erwähnten Alternativen (vgl. \cite[1]{ofn2015}). Er bleibt allerdings abstrakt genug, um Algorithmen (im Gegensatz zu vielen Programmiersprachen) unabhängig von Hardware und im Kontext unwichtigen Formalismen diverser Programmiersprachen darstellen zu können (vgl. \cite[1]{bem1958}). Die Darstellung durch Pseudocode macht allerdings dennoch ein erstes Verständnis des beschriebenen Algorithmus üblicherweise schwerer als andere Methoden, bleibt allerdings durch die Nähe zu möglichen konkreten Implementationen unmissverständlicher (vgl. \cite[147]{zob2015}). Aus letzterem Grund wird in der folgenden Arbeit Pseudocode zur Algorithmusbeschreibung verwendet.

\section{Pseudocode}
\label{sec:alg-spec-pseudo}

Es gibt nicht \emph{eine bestimmte} Pseudocode-Notation.

Programmiersprachen wie beispielsweise C++17 sind versioniert und standardisiert (vgl. \cite{ISO-C++17}), für Pseudocode gibt es keinen derart weit verbreiteten (oder wohldefinierten) Standard (vgl. \cite{kni1996}). Eine derartige Standardisierung wäre allerdings auch entgegen der Zielsetzung: Die Freiheit, Pseudocode an den gegebenen Kontext anzupassen, ist eine seiner großen Stärken. So wird zwar in einem Gutteil der in dieser Arbeit zitierten Werke Pseudocode verwendet (sofern im Kontext erfordert), konkrete Ausprägungen variieren jedoch in Syntax und Semantik\footnotemark.

\footnotetext{Vgl. beispielsweise die diversen Definitionen der verwendeten Pseudocode-Varianten in \cite[5]{hsr1997}, \cite[20]{clrs2001} und \cite[33]{ahu1974}, die jeweils an den Verwendungskontext angepasst sind.}

Wie schon in \prettyref{sec:algorithm-specification-overview} erwähnt, wird für Beschreibungen von Algorithmen in dieser Arbeit überwiegend Pseudocode zum Einsatz kommen. Genauer werden die \emph{\enquote{pseudocode conventions}} aus \cite[20-22]{clrs2001} verwendet, folgend kurz zusammengefasst.

\begin{itemize}
    \item Namen von Algorithmen sind in \text{Kapitälchen} gesetzt. Namen von Variablen sind \id{kursiv} gesetzt und üblicherweise einzelne Buchstaben ($i$, $n$, \ldots).

    \item Die \enquote{Blockstruktur} ist durch Einrückung vorgegeben. Beispielsweise beginnt der Körper der \textbf{while}-Schleife aus \prettyref{fig:pseudocode-algorithm-specification} auf \prettyref{ln:euclid-while-begin} und endet auf \prettyref{ln:euclid-while-end}.
    
    \item Die Konstrukte \textbf{while} und \textbf{for} sind ähnlich interpretierbar wie jene in Sprachen wie C oder Java, wobei eine Vereinfachung der in dieser Sprachgruppe üblichen Syntax der \textbf{for}-Schleife vorgenommen wurde.

    \begin{minipage}[t]{0.48\linewidth}
        \begin{codebox}
            \li $i \gets 2$
            \li \While $i < 100$
            \li     \Do
                        \Comment Körper der Schleife
            \li         $i \gets i^2$
            \li     \End                            \label{ln:pseudocode-example-while-end}
        \end{codebox}
        
        Während $i$ kleiner als 100 ist, wird der Körper der Schleife ausgeführt. Nach \prettyref{ln:pseudocode-example-while-end} gilt $i = 256$ 
    \end{minipage}
    \hfill
    \begin{minipage}[t]{0.48\linewidth}
        \begin{codebox}
            \li \For $j \gets 0$ \To $10$
            \li     \Do
                        \Comment Körper der Schleife
                    \End
        \end{codebox}

        Hier ist $j$ der \enquote{Schleifenzähler} und 10 der Endwert. Der Zähler wird um 1 erhöht, bis er \emph{gleich} dem Endwert ist.
    \end{minipage}

    \item Ebenso ähnlich zu Programmiersprachen wie C, \ldots\ ist die Verwendung von $=$ mit der Bedeutung einer \emph{Zuweisung} und $==$ mit der Beudeutung eines \emph{Vergleichs} (wie $<$ oder $\leq$).

    \item Wie aus den vorhergehenden Beispielen hervorgeht, deutet das Symbol \enquote{\Comment} darauf hin, dass der Rest der Zeile als Kommentar zu verstehen ist.
    
    \item Elemente eines \emph{Array} (Datenfeld) $A$ werden durch $A[i]$ abgerufen, wobei $i$ der Index des Elements ist. Arrays werden im Gegensatz zu vielen üblichen Pro\-grammier\-sprachen beginnend mit 1 indiziert und befüllt -- $A[1]$ ist das erste Element in einem Array $A$.
    
    Die Schreibweise $A[i \twodots j]$ wird verwendet um einen \enquote{Ausschnitt} bzw. ein \enquote{\emph{Subarray}} eines Arrays darzustellen. So steht $A[1 \twodots j]$ für das Subarray von $A$ welches (für $j > 2$) die $j$ Elemente $A[1]$, $A[2]$, \ldots $A[j]$ beinhaltet.
    
    Die Länge eines Arrays kann durch $\attrib{A}{length}$ abgerufen werden. (In C müsste hierfür eine separate Variable geführt werden.)

    \item Manchmal werden Schritte auch nur in natürlicher Sprache beschrieben, wenn dadurch die Verständlichkeit gefördert wird und dennoch die Eindeutigkeit der Schritte gegeben ist.

    So wird beispielsweise, wenn angebracht, \enquote{tausche $A[i]$ und $A[j]$ aus} anstelle von
    \begin{codebox}
        \li $t \gets A[i]$
        \li $A[i] \gets A[j]$
        \li $A[j] \gets t$
    \end{codebox}
    geschrieben.
    
    \item Eine \Return-Anweisung überträgt die Kontrolle und einen optionalen, auf die Anweisung folgenden Wert zurück zur Aufrufstelle.

    \ifndef{vwarelease}{\emph{Hier wäre ein Beispiel angebracht, vielleicht mit Rekursion?}}{}
\end{itemize}

\section{Sortieralgorithmen}
\label{sec:sorting-algorithms-definition}

Ein Sortieralgorithmus ist ein Algorithmus, der eine Eingabemenge $A$ in eine sortierte Permutation dieser Menge überführt, und als Ausgabemenge zurückgibt. Genauer muss bei einer Eingabemenge mit den $n$ Elementen $a_1, a_2, \ldots, a_n$ gelten, dass für alle Elemente $a'_1, a'_2, \ldots, a'_n$ der Ausgabemenge die Ordnungsrelation $a'_1 \leq a'_2 \leq \cdots \leq a'_n$ gilt (vgl. \cite[4]{taocp3}).

Ein ikonischer Sortieralgorithmus ist der \text{Insertion-Sort} (vgl. \cite[74]{taocp3}). Jedes Element wird einzeln betrachtet und in eine (wachsende) sortierte \enquote{Teilliste} \emph{eingefügt}. Dieser Algorithmus wird in \prettyref{sec:alg-insertion} (siehe insbesondere \prettyref{fig:insertions-example}) genauer behandelt, für ein grundlegendes Verständnis der Arbeitsweise von Algorithmen wird es jedoch hilfreich sein, ihn auch jetzt schon etwas genauer zu betrachten.

\begin{codebox}
    \Procname{$\text{Insertionsort}(A)$}
    \li \For $j \gets 2$ \To $\attrib{A}{length}$\label{ln:insertions-for-begin}
    \li     \Do\label{ln:ins-for-body-begin}
                $\id{key} \gets A[j]$\label{ln:ins-intermediary}
    \li         \Comment Insert $A[j]$ into the sorted sequence $A[1 \twodots j-1]$.
    \li         $i \gets j-1$\label{ln:ins-initial-i}
    \li         \While $i > 0$ and $A[i] > \id{key}$ \label{ln:ins-inner-while}
    \li             \Do
                        $A[i+1] \gets A[i]$\label{ln:ins-value-move}
    \li                 $i \gets i-1$
                    \End
    \li         $A[i+1] \gets \id{key}$\label{ln:ins-insertion}
            \End\label{ln:insertions-for-end}
\end{codebox}

Bevor ein Element $A[j]$ betrachtet wird besteht die Annahme, dass die Elemente $A[1], \ldots, A[j - 1]$ sortiert sind. (Aus diesem Grund beginnt die \For-Schleife in \prettyref{ln:insertions-for-begin} mit $j = 2$ und nicht etwa mit $j = 1$; das Subarray $A[1 \twodots j - 1]$ hat für $j = 2$ nur ein Element und ist somit immer \enquote{sortiert}.)

Jedes Element $A[2 \twodots \attrib{A}{length}]$ wird nun einzeln betrachtet und in $A[1 \twodots j - 1]$ \enquote{einsortiert}. Um ein Element derart in die Teilliste einzufügen zu können, muss
\begin{enumerate}[nosep, label=(\alph*)]
    \item der neue Index des einzufügenden Elements gefunden werden.
    \item Platz für das einzufügende Element geschaffen werden.
\end{enumerate}
Dies passiert in der \While-Schleife (beginnend mit \prettyref{ln:ins-inner-while}). Sie iteriert vorerst über alle möglichen Indices $i$, absteigend und beginnend mit dem höchstmöglichen ($i = j - 1$, \prettyref{ln:ins-initial-i}) bis zum kleinstmöglichen $i = 1$. Für jedes $i$ wird das Element $A[i]$ \enquote{nach oben}, zu $A[i + 1]$ verschoben (\prettyref{ln:ins-value-move}) -- somit wird Platz für das einzufügende Element geschaffen. 

Im Zuge dieser Verschiebung wird anfangs das Element $A[j]$ überschrieben, nachdem bei der ersten Iteration gilt, dass $j = i + 1$. Aus diesem Grund wird der Wert von $A[j]$ zu Beginn der äußeren Schleife in der Variable $\id{key}$ zwischengespeichert (\prettyref{ln:ins-intermediary}).

Gilt nach einer Verschiebung $A[i] \leq \id{key}$ (bzw. gilt also $A[i] > \id{key}$, \prettyref{ln:ins-inner-while}, \emph{nicht}), so wurde der neue Index des einzufügenden Elements gefunden: Das Element ist nach $A[i]$, also am Index $i + 1$ einzufügen (\prettyref{ln:ins-insertion}).

Nach Beendigung der \For-Schleife wurde die Eingabemenge vollständig sortiert.

\section{Effizienz}
\label{sec:algo-effizienz}

Die Effizienz eines Algorithmus ist bestimmt durch seinen Verbrauch von Ressourcen (vgl. \cite{rp2013}). In der folgenden Arbeit wird zwischen der praktischen Effizienz und der theoretischen Effizienz eines Algorithmus unterschieden. Erstere ist empirisch zu ermitteln (siehe \prettyref{cha:praktische-effizienz}), letztere wird durch mathematische Analyse bestimmt (siehe \prettyref{cha:asymptotic-analysis}).

Diese Unterscheidung geschieht in der Literatur üblicherweise implizit. So behandeln beispielsweise etwa \cite{mcg2012}, \cite{llx2012}, \cite{rp2013} und \cite{hsr1997} allesamt in irgendeiner Form die \enquote{Effizienz von Algorithmen}. \cite{mcg2012} behandelt sowohl theoretische als auch praktische Effizienz unter dem selben Namen, abhängig vom Kontext. \cite{llx2012} behandelt ausschließlich praktische Effizienz, nennt diese allerdings nur Effizienz. Als Gegenbeispiel behandeln \cite{rp2013} und \cite{hsr1997} nur theoretische Effizienz, nennen diese aber auch nur Effizienz.

Wie der Großteil vergleichbarer Analysen (vgl. bspw. \cite[23]{clrs2001} und \cite[58]{sha2011}) beschäftigt sich diese Arbeit mit der Analyse der von einem Algorithmus als Ressource verbrauchte Zeit, im Folgenden auch als \enquote{Laufzeit} bezeichnet. Wenn nicht näher angegeben, beziehen sich die Begriffe praktische und theoretische Effizenz immer auf die Laufzeit.

Üblicherweise ist die Effizienz eines Algorithmus proportional zur Größe der jeweiligen Eingangsmenge, deshalb wird die Effizienz im Folgenden (und im Allgemeinen auch in der Literatur\footnote{Alle Werke, die in dieser Arbeit zitiert werden und sich mit der Effizienz beziehungsweise Komplexität von Algorithmen beschäftigen, stellen diese in Abhängigkeit von einer Eingabegröße (auch: Problemgröße) dar.}) als Funktion $T$ in Abhängigkeit der \enquote{Eingabegröße} $n$ dargestellt, also als $T(n)$.

Ist die Effizienz eines Algorithmus zu ermitteln, gilt es also, die Zeit, die dieser benötigt, zumindest annähernd als Funktion in Abhängigkeit von $n$ darzustellen.
