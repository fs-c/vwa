\chapter{Vergleich der theoretischen und der praktischen Effizienzen}
\label{cha:vergleich}

\section{Konkrete Ermittlung der Daten}

Um den verwendeten Datenstamm zu generieren wird ein einzelnes Skript verwendet, welches Aufrufe an das benchmark-Programm und die Ablage und Weiterverarbeitung der Daten automatisiert. Somit wird eine Reproduktion der Daten erleichtert, es gilt nur einen einzelnen Befehl auszuführen. Das Skript ist in \crScriptsGenerate\ abgelegt\footnote{In selbigem Ordner ist auch eine README-Datei zu finden welche die Bedienung des Skripts näher erläutert.}.

Zur Generierung der Daten wurde durchgehend ein ThinkPad T480s (Modellnummer 20L8S02E00) verwendet, auf dem ein Intel Core i7-8550U (8 Threads $\times$ 1.80 GHz Base bzw. 4.00 GHz Turbo) und 16 GB DDR4 RAM (2400 MT/s mit NVMe SSD als Swap) verbaut sind.

\section{Approximierbarkeit der Effizienzen}

\ifndef{vwarelease}{\begin{center}
    \emph{Der Übersicht halber als Notiz: Effizienzen}

	\vspace{0.4cm}
	\begin{tabular}{c | c  c}\toprule
		 & Best & Worst \\\midrule
		Insertion & O($n$) & O($n^2$) \\
		Quick & O($n \log n$) & O($n^2$) \\
		Heap & O($n \log n$) & O($n \log n$)\\
		Merge & O($n \log n$) & O($n \log n$) \\\bottomrule
	\end{tabular}
	\vspace{0.4cm}
\end{center}}{}

\begin{figure}[p]
	\centering

	\makebox[\textwidth][c]{\subfloat[Heap, sorted]{
		\begin{tikzpicture}
			\begin{axis}[default, width=0.65\linewidth, height=4cm]
				\addplot [black] table	{data/supplementary/comparison/heap_sorted_best_diff};
			\end{axis}
		\end{tikzpicture}
	}\hfil
	\subfloat[Insertion, sorted]{
		\begin{tikzpicture}
			\begin{axis}[default, width=0.65\linewidth, height=4cm]
				\addplot [black] table	{data/supplementary/comparison/insertion_sorted_best_diff};
			\end{axis}
		\end{tikzpicture}
	}}

	\bigskip
	\makebox[\textwidth][c]{\subfloat[Merge, sorted]{
		\begin{tikzpicture}
			\begin{axis}[default, width=0.65\linewidth, height=4cm]
				\addplot [black] table	{data/supplementary/comparison/merge_sorted_best_diff};
			\end{axis}
		\end{tikzpicture}
	}\hfil
	\subfloat[Quick, sorted]{
		\begin{tikzpicture}
			\begin{axis}[default, width=0.65\linewidth, height=4cm]
				\addplot [black] table	{data/supplementary/comparison/quick_sorted_best_diff};
			\end{axis}
		\end{tikzpicture}
	}}

	\bigskip
	\makebox[\textwidth][c]{\subfloat[Heap, inverted]{
		\begin{tikzpicture}
			\begin{axis}[default, width=0.65\linewidth, height=4cm]
				\addplot [black] table	{data/supplementary/comparison/heap_inverted_worst_diff};
			\end{axis}
		\end{tikzpicture}
	}\hfil
	\subfloat[Insertion, inverted]{
		\begin{tikzpicture}
			\begin{axis}[default, width=0.65\linewidth, height=4cm]
				\addplot [black] table	{data/supplementary/comparison/insertion_inverted_worst_diff};
			\end{axis}
		\end{tikzpicture}
	}}

	\bigskip
	\makebox[\textwidth][c]{\subfloat[Merge, inverted]{
		\begin{tikzpicture}
			\begin{axis}[default, width=0.65\linewidth, height=4cm]
				\addplot [black] table	{data/supplementary/comparison/merge_inverted_worst_diff};
			\end{axis}
		\end{tikzpicture}
	}\hfil
	\subfloat[Quick, inverted]{
		\begin{tikzpicture}
			\begin{axis}[default, width=0.65\linewidth, height=4cm]
				\addplot [black] table	{data/supplementary/comparison/quick_inverted_worst_diff};
			\end{axis}
		\end{tikzpicture}
	}}
	\caption{Differenz zwischen erwarteter und tatsächlicher Zeit, die der jeweilige Algorithmus für das bearbeiten der jeweiligen Eingabemenge benötigt hat. Negative Werte sind schneller, positive langsamer als die Projektion.}
	\label{fig:approximation-delta}
\end{figure}

\section{Der \enquote{beste Algorithmus}}
