\chapter{Vergleich der theoretischen und der praktischen Effizienzen}
\label{cha:vergleich}

\section{Konkrete Ermittlung der praktischen Effizienz}

\emph{Folgende Commands werden für das generieren der Daten verwendet. Zu erweitern.}

\begin{lstlisting}
benchmark -r -a 16 -t linear -s 16 -c 16 -o ./linear/xs       # 2^4
benchmark -r -a 16 -t linear -s 256 -c 128 -o ./linear/sm      # 2^8
benchmark -r -a 16 -t linear -s 262144 -c 128 -o ./linear/md   # 2^18

benchmark -r -a 16 -t quadratic -s 262144 -c 128 -o ./quadratic/coarse
benchmark -r -a 16 -t quadratic -s 262144 -c 512 -o ./quadratic/fine
\end{lstlisting}

\section{Approximierbarkeit der Effizienzen}

\ifndef{vwarelease}{\begin{center}
    \emph{Der Übersicht halber als Notiz: Effizienzen}

	\vspace{0.4cm}
	\begin{tabular}{c | c  c}\toprule
		 & Best & Worst \\\midrule
		Insertion & O($n$) & O($n^2$) \\
		Quick & O($n \log n$) & O($n^2$) \\
		Heap & O($n \log n$) & O($n \log n$)\\
		Merge & O($n \log n$) & O($n \log n$) \\\bottomrule
	\end{tabular}
	\vspace{0.4cm}
\end{center}}{}

\begin{figure}[p]
	\centering

	\makebox[\textwidth][c]{\subfloat[Heap, sorted]{
		\begin{tikzpicture}
			\begin{axis}[default, width=0.65\linewidth, height=4cm]
				\addplot [black] table	{data/supplementary/comparison/heap_sorted_best_diff};
			\end{axis}
		\end{tikzpicture}
	}\hfil
	\subfloat[Insertion, sorted]{
		\begin{tikzpicture}
			\begin{axis}[default, width=0.65\linewidth, height=4cm]
				\addplot [black] table	{data/supplementary/comparison/insertion_sorted_best_diff};
			\end{axis}
		\end{tikzpicture}
	}}

	\bigskip
	\makebox[\textwidth][c]{\subfloat[Merge, sorted]{
		\begin{tikzpicture}
			\begin{axis}[default, width=0.65\linewidth, height=4cm]
				\addplot [black] table	{data/supplementary/comparison/merge_sorted_best_diff};
			\end{axis}
		\end{tikzpicture}
	}\hfil
	\subfloat[Quick, sorted]{
		\begin{tikzpicture}
			\begin{axis}[default, width=0.65\linewidth, height=4cm]
				\addplot [black] table	{data/supplementary/comparison/quick_sorted_best_diff};
			\end{axis}
		\end{tikzpicture}
	}}

	\bigskip
	\makebox[\textwidth][c]{\subfloat[Heap, inverted]{
		\begin{tikzpicture}
			\begin{axis}[default, width=0.65\linewidth, height=4cm]
				\addplot [black] table	{data/supplementary/comparison/heap_inverted_worst_diff};
			\end{axis}
		\end{tikzpicture}
	}\hfil
	\subfloat[Insertion, inverted]{
		\begin{tikzpicture}
			\begin{axis}[default, width=0.65\linewidth, height=4cm]
				\addplot [black] table	{data/supplementary/comparison/insertion_inverted_worst_diff};
			\end{axis}
		\end{tikzpicture}
	}}

	\bigskip
	\makebox[\textwidth][c]{\subfloat[Merge, inverted]{
		\begin{tikzpicture}
			\begin{axis}[default, width=0.65\linewidth, height=4cm]
				\addplot [black] table	{data/supplementary/comparison/merge_inverted_worst_diff};
			\end{axis}
		\end{tikzpicture}
	}\hfil
	\subfloat[Quick, inverted]{
		\begin{tikzpicture}
			\begin{axis}[default, width=0.65\linewidth, height=4cm]
				\addplot [black] table	{data/supplementary/comparison/quick_inverted_worst_diff};
			\end{axis}
		\end{tikzpicture}
	}}
	\caption{Differenz zwischen erwarteter und tatsächlicher Zeit, die der jeweilige Algorithmus für das bearbeiten der jeweiligen Eingabemenge benötigt hat. Negative Werte sind schneller, positive langsamer als die Projektion.}
	\label{fig:approximation-delta}
\end{figure}

\section{Der \enquote{beste Algorithmus}}
