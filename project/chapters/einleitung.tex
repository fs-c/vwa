\chapter*{Einleitung}
% "Unnumber" chapters/sections/... have to be added to the table of contents manually
\addcontentsline{toc}{chapter}{Einleitung}

\ifndef{vwarelease}{
    \emph{Ein hochgestelltes \enquote{[?]} weist auf fehlende Quellen hin. In der \enquote{Release Version} ist es, wie dieser Absatz und die obenstehenden Hinweise, unsichtbar.}
}{}

%Vorerst ist es ausreichend, einen Algorithmus als eine schwarze Box (vgl. \cite{bun1963}) zu betrachten, die auf eine deterministische\footnote{Alle auftretenden Zustände sind definiert und reproduzierbar, der \enquote{nächste Schritt} ist immer eindeutig festgelegt (vgl. \cite[1]{baas2009}).} Art und Weise eine Eingangsmenge zu einer Ausgangsmenge überführt (\cite[5]{clrs2001}). Ein Sortieralgorithmus überführt die Elemente seiner Eingangsmenge zu einer geordneten Ausgangs\-menge.

%Die Effizienz eines Algorithmus ist bestimmt durch seinen Verbrauch von Ressourcen. Folgend wird zwischen \enquote{praktischer Effizienz}\nocit\ und \enquote{theoretischer Effizienz}\ eines Algorithmus unterschieden. Erstere ist empirisch zu ermitteln: Der Ressourcenverbrauch eines Algorithmus wird konkret gemessen. Letztere wird durch mathematische Analyse bestimmt.

%Damit können bereits die zwei Forschungsfragen dieser Arbeit formuliert werden:

\begin{itemize}
    \item Wie kann die \enquote{praktische-} und die \enquote{theoretische Effizienz} eines Algorithmus ermittelt werden?
    \item Wie verhält sich die \enquote{theoretische Effizienz} von ausgewählten Sortieralgorithmen zu der (zu ermittelnden) \enquote{praktischen Effizienz} jener Algorithmen?
\end{itemize}

%Wie der Großteil vergleichbarer Analysen (vgl. \cite[23]{clrs2001} und \cite[58]{sha2011} \ldots\nocit) beschäftigt sich diese Arbeit mit der Analyse der von einem Algorithmus als Ressource verbrauchte Zeit, im Folgenden auch als \enquote{Laufzeit} bezeichnet. Wenn nicht näher angegeben beziehen sich die Begriffe \enquote{praktische}- und \enquote{theoretische Effizenz} immer auf die Laufzeit.

Für konkrete Implementationen und Quellcode-Beispiele wird die Programmiersprache C++ (\cite{ISO-C++17}, \enquote{C++17}) verwendet.

$\log x$ ist als $\log_{10} x$ bzw. als \enquote{Logarithmus zur Basis 10 von $x$} zu verstehen. 

\paragraph{Notizen}

Nach \cite{mcg2012}[3] sind die hauptsächlichen Nachteile der theoretischen Analyse
\begin{itemize}
    \item mangelnde Spezifität (ein in Pseudocode implementierter Algorithmus könnte weit von einer funktionierenden Implementation entfernt sein) und
    \item die oftmals stark vereinfachten \enquote{Berechnungsmodelle}.
\end{itemize}


