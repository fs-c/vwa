\chapter*{Einleitung}
% "Unnumber" chapters/sections/... have to be added to the table of contents manually
\addcontentsline{toc}{chapter}{Einleitung}

\ifndef{vwarelease}{
    \emph{Ein hochgestelltes \enquote{[?]} weist auf fehlende Quellen hin. In der \enquote{Release Version} ist es, wie dieser Absatz und die obenstehenden Hinweise (\emph{\enquote{Vorschau}}), unsichtbar.}

    \emph{Sind ganze Sätze (oder Absätze) kursiv gesetzt so sind sie ebenfalls nur in dieser Vorschauversion sichtbar.}
}{}

Die zwei primären Leitfragen dieser Arbeit sind:

\begin{itemize}
    \item Wie kann die \enquote{theoretische-} und die \enquote{praktische Effizienz} eines Algorithmus ermittelt werden?
    \item Wie verhält sich die \enquote{theoretische Effizienz} von ausgewählten Sortieralgorithmen zu der (zu ermittelnden) \enquote{praktischen Effizienz} jener Algorithmen?
\end{itemize}

Diese zwei Fragen sind der Kern dieser Arbeit, sie werden in \prettyref{cha:asymptotic-analysis}, \prettyref{cha:praktische-effizienz} und \prettyref{cha:algorithmen} behandelt. Als Basis der Erarbeitung dieser Fragestellungen dient \prettyref{cha:definition-funktion-algorithmen}.

$\log x$ ist als $\log_{e} x$ bzw. als \enquote{Logarithmus zur Basis $e$ von $x$} zu verstehen. \ifndef{vwarelease}{\emph{Dieser Hinweis ist in einer candidate release Phase als Fußnute der ersten Verwendung von $\log \ldots$ beizufügen.}}{}
