% For proper support of the German language.
\usepackage[utf8]{inputenc}
\usepackage[T1]{fontenc}
\usepackage[ngerman]{babel}

% Custom commands, most importantly \ifndef.
% If #1 is defined insert #2, otherwise insert #3.
% \ifndef{somecommand}{\emph{not defined}}{\emph{defined}}
\makeatletter
\newcommand{\ifndef}[3]{%
    \@ifundefined{#1}{%
        #2
    }{
        #3
    }%
}%
\makeatother

% If we are in release mode insert #1, otherweise insert #2 (see \ifndef). 
\makeatletter
\ifndef{vwarelease}{
    \newcommand{\nocit}{%
        \textsuperscript{[?]}%
        \ 
        \@latex@warning{Missing source}%
    }%
}{
    \newcommand{\nocit}[]{%
        \empty%
        \@latex@warning{Ignored missing source}%
    }%
}
\makeatother

\newcommand{\floor}[1]{\left\lfloor #1 \right\rfloor}
\newcommand{\ceil}[1]{\left\lceil #1 \right\rceil}

\newcommand{\spaced}[1]{\medskip {\noindent #1} \medskip}

% Basically a reimplementation of Knuths algorithm specification
\usepackage{enumitem}
\newlist{algosteps}{enumerate}{1}
\newenvironment{algo}[2]{
    \newcommand{\algostep}[1][]{\item {[##1]}}
    \setlist[algosteps]{before=\smallskip, nosep, leftmargin=*, label=\textbf{#1\arabic*}, ref=#1\arabic*}
    \par\noindent\textbf{Algorithmus~#1}~(\textit{#2}).
}{
}

% Footnote without a marker
\newcommand\freefootnote[1]{%
    \begingroup
    \renewcommand\thefootnote{}\footnote{#1}%
    \addtocounter{footnote}{-1}%
    \endgroup
}


% References to online code samples.
\newcommand{\crBenchmark}{src/benchmark.h}
\newcommand{\crBenchmarkImpl}{src/benchmark.cpp}

\newcommand{\crMain}{src/main.cpp}

\newcommand{\crScriptsGenerate}{scripts/data/generate.js}
\newcommand{\crScriptsReciprocalApprox}{scripts/data/reciprocal-approx.js}

\newcommand{\crData}{project/data}
\newcommand{\crDataCanon}{\crData/canon}
\newcommand{\crDataSup}{\crData/canon}
\newcommand{\crDataSupComparison}{\crDataSup/comparison}


% Because good typography is cool
\usepackage{microtype}

% For \enquote.
\usepackage{csquotes}

% For convenient math typesetting
\usepackage{amsmath}
\usepackage{amssymb}

% To generate a proper appendix. title adds "Appendix" to appendix titles, titletoc does the same to appendix titles in the ToC.
\usepackage[titletoc,title]{appendix}

% Bold \ttfamily versions
\usepackage{lmodern}

% For \lstlisting, provides colored code samples.
% TODO: Automatically broken lines are indented by x pixels, not with tabs.
\usepackage{listings}
\lstset{
    language=C++,
    % Basic styling
    tabsize=4,
    breaklines=true,
    numbers=left,
    numberstyle=\tiny,
    keywordstyle=\bfseries,
    columns=flexible,
    % For consistency with figures and tables
    captionpos=b,
    % Used to escape commands in the environment
    escapechar=|,
    mathescape=true,
}
\lstset{
    literate={ö}{{\"o}}1
        {ä}{{\"a}}1
        {ü}{{\"u}}1
}
\lstdefinestyle{pseudo}{%
    language=pascal,
}

% For pseudocode like in clrs2001
\usepackage{clrscode3e}

% Because sometimes, multiple columns are nice
\usepackage{multicol}

% For the \prettyref command, which allows prefixes to references to be defined in a central place (here).
\usepackage{prettyref}
\newrefformat{lst}{Listing \ref{#1}}
\newrefformat{ln}{Zeile \ref{#1}}
\newrefformat{cha}{Kapitel \ref{#1}}
\newrefformat{sec}{Abschnitt \ref{#1}}
\newrefformat{tab}{Tabelle \ref{#1}}
\newrefformat{fig}{Abbildung \ref{#1}}
\newrefformat{subfig}{Abbildung \ref{#1}}
\newrefformat{itm}{Punkt \ref{#1}}

% Horizontal lists, among many other things.
\usepackage{enumitem}[shortlabels, inline]

% Well... it's tikz. Used only for little illustrations, at the moment.
\usepackage{tikz}
% For horizontal arrow tips on vertical lines
\usepgflibrary{arrows.meta}
% Used to draw all plots/graphs
\usepackage{pgfplots}
\pgfplotsset{compat=1.16}
% Used to draw trees
\usepackage[edges]{forest}

% To allow multiple subfigures inside a figure.
\usepackage[labelformat=simple]{subcaption}
% Wrap references to, and labels of subfigures in parentheses (i.e. "(a) description of the subfigure" and "see Figure 1.2(a)")
\renewcommand\thesubfigure{ (\alph{subfigure})}

% For proper table formatting
\usepackage{booktabs}

% Recommended for biblatex.
\usepackage{xpatch}
% For coherent, managed citations.
\usepackage[style=authoryear, citereset=subsection+, maxbibnames=99, doi=false,isbn=false,url=false,eprint=false]{biblatex}
% Add all .bib files in the bibliography folder.
\addbibresource{bibliography/main.bib}
% Add a comma between a publications author name and year to comply with the guideline.
\DeclareDelimFormat{nameyeardelim}{\addcomma\space}

% Fill all horizontal space with dots, in the table of contents
\usepackage{tocloft}
\renewcommand{\cftchapleader}{\cftdotfill{\cftdotsep}}

% For \lipsum, which generates blind text
\ifndef{vwarelease}{
    \usepackage{lipsum}
}{}

% For captions which are placed on the side of a figure (SCfigure environment)
\usepackage{sidecap}

% Used to style a whole column of a table
\usepackage{array}

% Customise the footer and header; add note about development mode in header.
\usepackage{fancyhdr}
\fancyhf{}
\pagestyle{fancyplain}

\renewcommand{\headrulewidth}{0pt}

% For \pageref{LastPage}
\usepackage{lastpage}
\rfoot{\protect \thepage\ von\ \protect \pageref{LastPage}}

% For uppercase text, to space it out a little more
\usepackage{soul}
\sodef\so{}{.14em}{.4em plus.1em minus .1em}{.4em plus.1em minus .1em}

% To change line spacing
\usepackage{setspace}

% For named references
\usepackage{nameref}

\ifndef{vwarelease}{%
    \setlength{\headheight}{15.2pt}%

    \chead{Vorschau}%
}{}

% All cross-referenced elements become hyperlinked (e.g. the table of contents). Also used for \url. This should be the last import.
\usepackage{hyperref}
