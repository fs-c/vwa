\begin{table}[htp]
    \makebox[\textwidth][c]{\begin{tabular}{c >{\em}c | c >{\em}c | c >{\em}c | c >{\em}c}
        \toprule
        $n$ & $\frac{n}{n}$ & $n^2$ & $\frac{n}{n^2}$ & $\log n$ & $\frac{n}{\log n}$ & $n \log n$ & $\frac{n}{n \log n}$ \\\midrule
        1 & 1 & 1 & 1 & 0 & & 0 & \\
        4 & 1 & 16 & 0.25 & 1.38629436 & 2.88539008 & 5.54517744 & 0.72134752 \\
        16 & 1 & 256 & 0.0625 & 2.77258872 & 5.77078016 & 44.3614195 & 0.36067376 \\
        64 & 1 & 4096 & 0.015625 & 4.15888308 & 15.3887471 & 266.168517 & 0.24044917 \\
        256 & 1 & 65536 & 0.00390625 & 5.54517744 & 46.1662413 & 1419.56542 & 0.18033688 \\
        1024 & 1 & 1048576 & 0.00097656 & 6.93147180 & 147.731972 & 7097.82712 & 0.14426950 \\
        4096 & 1 & 16777216 & 0.00024414 & 8.31776616 & 492.439907 & 34069.5702 & 0.12022458 \\
        16384 & 1 & 268435456 & 0.00006103 & 9.70406052 & 1688.36539 & 158991.327 & 0.10304964 \\
        65536 & 1 & 4294967296 & 0.00001525 & 11.0903548 & 5909.27888 & 726817.498 & 0.09016844 \\
        \bottomrule       
    \end{tabular}}
    \caption{Werte der Funktionen $f(n) = n$, $f(n) = n^2$, $f(n) = \log n$ und $f(n) = n \log n$ mit nebengestelltem, \emph{kursiv gesetztem}, Quotient aus dem jeweiligem Funktionswert und $n$. Leerstehende Felder repräsentieren ein undefiniertes Ergebnis. Es gilt $n = 2^0, 2^2, 2^4, \ldots, 2^16$.}
    \label{tab:common-growth-rates}
\end{table}
