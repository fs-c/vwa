\begin{figure}[ht]
    \subcaptionbox{\emph{Quick sort} auf sehr kleinen, sortierten Eingabemengen; links ist die Größe der Eingabemenge, rechts die Zeit in Mikrosekunden.\label{fig:funkterm-beispieldaten}}[0.45\textwidth]{%
        \begin{minipage}{0.2\textwidth}
            \begin{tabular}{cc}
                \toprule
                Größe & Zeit \\
                \midrule
                1 & 2.836\\
                2 & 3.16\\
                3 & 2.535\\
                4 & 3.346\\
                5 & 4.737\\
                6 & 4.925\\
                7 & 5.643\\
                9 & 5.337\\
                \bottomrule
            \end{tabular}
        \end{minipage} \hfill
        \begin{minipage}{0.2\textwidth}
            \begin{tabular}{cc}
                \toprule
                 &  \\
                \midrule
                9 & 6.23\\
                10& 7.363\\
                11& 8.583\\
                12& 9.094\\
                13& 9.499\\
                14& 10.103\\
                15& 10.548\\
                16& 11.757\\
                \bottomrule
            \end{tabular}
        \end{minipage}
        \vspace{1.2em}
    }%
    \hfill
    \subcaptionbox{Graph der Daten in \subref{fig:funkterm-beispieldaten}.\label{fig:funkterm-beispielgraph}}{%
        \begin{tikzpicture}
        \begin{axis}[
            xlabel={Größe},
            ylabel={Zeit},
            grid=major,
            width=0.5\textwidth,
        ]
            \addplot [] table {data/xs/quick_sorted};
        \end{axis}
        \end{tikzpicture}
    }%
    \caption{Demonstration des Ausgabeformats aus \prettyref{lst:benchmark} mit daraus generiertem Graphen.\label{fig:function-determination-raw-data}}
\end{figure}