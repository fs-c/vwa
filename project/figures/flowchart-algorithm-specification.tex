\begin{figure}[hbt]
    \centering
    \makebox[\textwidth][c]{
        \begin{tikzpicture}[
            auto,
            node distance=1.5cm,
            algostep/.style={rectangle, minimum size=6mm, draw=black, inner xsep=0.5em}
        ]
            \node (s1) [algostep] {\textbf{\ref{step:euklid1}}. Finde den Rest.};
            \node (s2) [algostep, rounded corners=3mm, right=of s1] {\textbf{\ref{step:euklid2}}. Ist er Null?};
            \node (s3) [algostep, right=of s2] {\textbf{\ref{step:euklid3}}. Verringere.};

            \node (a1) [above=0.7 cm of s1] {$\{n, m\}$};
            \node (a2) [below=0.7 cm of s2] {$n$};

            \draw [->]
                (s1) edge (s2)
                (s2) edge node {Nein} (s3)
                (a1) edge (s1)
                (s2) edge node {Ja} (a2);
            
            \draw [->] (s3.north) -- +(0,0.5) -| (s1);
        \end{tikzpicture}
    }
    \caption{Algorithmusbeschreibung in Form eines Flowcharts. Beispiel in abgeänderter Form aus \cite[3]{taocp1}, Fig. 1 entnommen.}
    \label{fig:flowchart-algorithm-specification}
\end{figure}
