\begin{figure}[htbp]
    \centering
    \begin{subfigure}{0.75\linewidth}
        \begin{algo}{E}{Euklids Algorithmus}
            Gegeben seien zwei positive ganze Zahlen $m$ und $n$. Der \emph{größte gemeinsame Teiler}, also die größte positive ganze Zahl welche sowohl $m$ als auch $n$ gerade teilt, ist zu ermitteln.
        
            \begin{algosteps}
                \algostep[Ermittle den Rest.] Dividiere $m$ durch $n$, $r$ ist der Rest. (Es gilt nun also $0 \leq r < n$.)\label{step:euklid1}
        
                \algostep[Ist er Null?] Gilt $r = 0$ so terminiert der Algorithmus, $n$ ist die Lösung.\label{step:euklid2}
                
                \algostep[Verringern.] Setze $m \leftarrow n$, $n \leftarrow r$ und gehe zurück zu Schritt \ref{step:euklid1}\label{step:euklid3}.
            \end{algosteps}
        \end{algo}
    
        \subcaption{Beschreibung nach \citeauthor{taocp1}, u. A. verwendet in \cite{taocp1} und \cite{taocp3}. Beispiel entnommen aus \cite[2]{taocp1}, Algorithm E. \label{subfig:knuth-algorithm-specification}}
    \end{subfigure}

    \begin{subfigure}[t]{0.75\linewidth}
        \medskip
        \centering
        \begin{tikzpicture}[
            auto,
            node distance=0.75cm,
            algostep/.style={rectangle, minimum size=6mm, draw=black, inner xsep=0.5em}
        ]
            \node (s1) [algostep] {\textbf{\ref{step:euklid1}}. Dividiere $m$ durch $n$.};
            \node (s2) [algostep, rounded corners=2mm, below=of s1] {\textbf{\ref{step:euklid2}}. Ist der Rest $r$ Null?};
            \node (s3) [algostep, below=of s2] {\textbf{\ref{step:euklid3}}. Setze $m$ zu $n$ und $n$ zu $r$.};

            \node (a1) [above=of s1] {$\{n, m\}$};
            \node (a2) [right=of s2] {$n$};

            \draw [->]
                (a1) edge (s1)
                (s2) edge node {Ja} (a2)
                (s1) edge (s2)
                (s2) edge node {Nein} (s3);
            
            \draw [->] (s3.west) -- +(-0.5,0) |- (s1);
        \end{tikzpicture}

        \subcaption{Beschreibung in Form eines Flowcharts. Beispiel in abgeänderter Form aus \cite[3]{taocp1}, Fig. 1 entnommen.\label{subfig:flowchart-algorithm-specification}}
    \end{subfigure}
    \hfill
    \begin{subfigure}[t]{0.75\linewidth}
        \medskip
        \centering
        \begin{codebox}
            \Procname{$\text{Euklid}(n, m)$}
            \li \While $(r \gets n \mod m) \neq 0$
            \li     \Do
                        $m \gets n$                 \label{ln:euclid-while-begin}
            \li         $n \gets r$                 \label{ln:euclid-while-end}
                    \End
            \li \Return $n$                         \label{ln:euclid-return}
        \end{codebox}
        \subcaption{Beschreibung durch \emph{Pseudocode} wie er auch in der folgenden Arbeit verwendet wird.\label{subfig:pseudocode-algorithm-specification}}
    \end{subfigure}
    \caption{Diverse Arten der Algorithmusbeschreibung am Beispiel des euklidischen Algorithmus.}
    \label{fig:algorithm-notation-examples}
\end{figure}
