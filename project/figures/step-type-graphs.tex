\begin{figure}[h]
	\centering
	\begin{subfigure}[T]{0.455\textwidth}
		\begin{tikzpicture}
			\begin{axis}[
				default, width=0.95\textwidth,
				xmin=0, ymin=0, xmax=32,
				domain=0:32,
				legend pos=north west,
				xlabel={Index},
				ylabel={Größe},
			]
				\addplot[black] {(262144 / 128) * x};
				\addlegendentry{$l$};
				\addplot[black, thick] {262144 / 128^2) * x^2};
				\addlegendentry{$q$};
			\end{axis}
		\end{tikzpicture}
	\end{subfigure}
	\hfill
	\begin{subfigure}[T]{0.455\textwidth}
		\begin{tikzpicture}
			\begin{axis}[
				default, width=0.95\textwidth,
				xmin=0, ymin=0, xmax=128,
				domain=0:128,
			]
				\addplot[black] {(262144 / 128) * x};
				\addplot[black, thick] {262144 / 128^2) * x^2};
			\end{axis}
		\end{tikzpicture}
	\end{subfigure}
	\caption{Graphen der Untermengengrößen bei einer gesamten Mengengröße von $2^{18}$, geteilt auf 128 Untermengen mit linearer ($l$) und quadratischer ($q$) Schrittgröße.}
	\label{fig:step-type-graphs}
\end{figure}
