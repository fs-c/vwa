\chapter{Implementationen}
\label{app:implementations}

\ifndef{vwarelease}{\emph{%
    Mögl. alternativer Titel: Algorithmusimplementationen. Listings sollten floating sein und labels und descriptions haben.
}}

\begin{lstlisting}[label=lst:sets, caption={Implementation von \enquote{Generatoren} für diverse Arten von Eingabemengen.}]
namespace sets {
	using set_t = std::vector<int>;
	using iterator_t = set_t::iterator;

	set_t sorted(const size_t size) {
		auto set = set_t(size);

		std::iota(set.begin(), set.end(), 1);

		return set;
	}

	set_t inverted(const size_t size) {
		auto set = set_t(size);

		std::iota(std::rbegin(set), std::rend(set), 1);

		return set;
	}

	set_t random(const size_t size) {
		auto set = sorted(size);

		utils::random_shuffle(set.begin(), set.end());

		return set;
	}

	...
}
\end{lstlisting}

\begin{lstlisting}[float, label=lst:insertion, caption={Implementation des \emph{insertion sort}.}]
template <class I, class P = std::less<>>
void insertion(I first, I last, P cmp = P{}) {
	for (auto it = first; it != last; ++it) {
		auto const insertion = std::upper_bound(first, it, *it, cmp);
		std::rotate(insertion, it, std::next(it));
	}
}
\end{lstlisting}

\begin{lstlisting}[float, label=lst:quick, caption={Implementation des \emph{quicksort}.}]
template <class I, class P = std::less<>>
void quick(I first, I last, P cmp = P{}) {
	auto const N = std::distance(first, last);
	if (N <= 1)
		return;

	auto const pivot = *std::next(first, N / 2);

	auto const middle1 = std::partition(first, last, [=](auto const &elem) {
		return cmp(elem, pivot); });
	auto const middle2 = std::partition(middle1, last, [=](auto const &elem) {
		return !cmp(pivot, elem); });

	quick(first, middle1, cmp);
	quick(middle2, last, cmp);
}
\end{lstlisting}

\begin{lstlisting}[float, label=lst:heap, caption={Implementation des \emph{heapsort}.}]
template<class RI, class P = std::less<>>
void heap(RI first, RI last, P cmp = P{}) {
    std::make_heap(first, last, cmp);
    std::sort_heap(first, last, cmp);
}
\end{lstlisting}

\begin{lstlisting}[float, label=lst:merge, caption={Implementation des \emph{merge sort}.}]
template <class BI, class P = std::less<>>
void merge(BI first, BI last, P cmp = P{}) {
	auto const N = std::distance(first, last);
	if (N <= 1)
		return;

	auto const middle = std::next(first, N / 2);

	merge(first, middle, cmp);
	merge(middle, last, cmp);

	std::inplace_merge(first, middle, last, cmp);
}
\end{lstlisting}

\begin{lstlisting}[float, label=lst:benchmark-write, caption={Implementation einer Funktion zum Speichern des Rückgabewerts von \lstinline{benchmark::run} aus \prettyref{lst:benchmark}.}]
void write(std::string sub_path, const char *algo_name, const char *set_name, timings_t timings) {
	std::filesystem::path file_path;

	if (sub_path.size())
		file_path += sub_path;

	file_path += "/";

	std::filesystem::create_directories(file_path);

	file_path += algo_name;
	file_path += "_";
	file_path += set_name;

	printf("writing %s\n", file_path.c_str());

	if (std::filesystem::exists(file_path)) {
		std::filesystem::remove(file_path);
	}

	std::ofstream os(file_path.c_str());

	for (const auto [n, t] : timings) {
		os << n << " " << t << "\n";
	}
}
\end{lstlisting}

\begin{lstlisting}[float, label=lst:random-shuffle, caption={Implementation eines Algorithmus zum zufälligen Durchmischen einer Eingabemenge unter verwendung des Mersenne-Twister Algorithmus.}]
template <class I>
void random_shuffle(I &&first, I &&last) {
	std::random_device rd;
	std::mt19937 g(rd());

	std::shuffle(first, last, g);
}
\end{lstlisting}